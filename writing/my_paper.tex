\documentclass{article}

% these packages let you do math
\usepackage{amsmath}
\usepackage{amssymb}

% we need these packages for fancy R tables
\usepackage{booktabs}
\usepackage{float}
\usepackage{colortbl}
\usepackage{xcolor}

% these packages play with the spacing/margins of the document. Uncomment the commands on lines 16 and 17 to see what they do.
\usepackage{a4wide}
\usepackage{setspace}
\usepackage{geometry}
\usepackage{parskip}
%\doublespacing
%\geometry{margin=1.5in}

% this package helps us with including images. Setting the graphics path makes it easier to refer to things in the \includegraphics command.
\usepackage{graphicx}
\graphicspath{ {../figures/} }

% make some hyperlinks using the \href command
\usepackage{hyperref}
\hypersetup{
    colorlinks=true,
    linkcolor=black,
    urlcolor=blue
}

% set the author, title, and date of the document. \maketitle adds it to the document.
\author{Xiaohan Wu}
\title{My Paper on NLSY97 Data of Incarceration Status}
\date{Spring 2022}

\begin{document}
\maketitle

\section{The First Section}



\begin{figure}[H]
    \begin{center}
        \includegraphics[width=.85\textwidth]{incarcerations_by_racegender}
    \end{center}
    \caption{Mean Number of Incarcerations in 2002 by Race and Gender (this is the LaTeX caption, not the ggplot title)}
    \label{fig:graph}
\end{figure}

\input{../tables/incarcerations_by_racegender.tex}

\input{../tables/regress_incarcerations_by_racegender.tex}

\end{document}
