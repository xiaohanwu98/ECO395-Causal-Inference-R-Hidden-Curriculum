\documentclass{article}

% these packages let you do math
\usepackage{amsmath}
\usepackage{amssymb}

% we need these packages for fancy R tables
\usepackage{booktabs}
\usepackage{float}
\usepackage{colortbl}
\usepackage{xcolor}

% these packages play with the spacing/margins of the document. Uncomment the commands on lines 16 and 17 to see what they do.
\usepackage{a4wide}
\usepackage{setspace}
\usepackage{geometry}
\usepackage{parskip}
%\doublespacing
%\geometry{margin=1.5in}

% this package helps us with including images. Setting the graphics path makes it easier to refer to things in the \includegraphics command.
\usepackage{graphicx}
\graphicspath{ {../figures/} }

% make some hyperlinks using the \href command
\usepackage{hyperref}
\hypersetup{
    colorlinks=true,
    linkcolor=black,
    urlcolor=blue
}

% set the author, title, and date of the document. \maketitle adds it to the document.
\author{Xiaohan Wu}
\title{Paper on NLSY97 Data of Incarceration Status}
\date{Spring 2022}

\begin{document}
\maketitle

\section{Introduction}

This report describes patterns in incarceration status by race and gender in the year 2002, using NLSY97 publicly available data. Before analyzing the dataset, the raw dataset was mined into a clean dataset with readable variables wanted.

The three variables to be analyzed are \texttt{race}, \texttt{gender}, and \texttt{total\_incarcerated}. The variable \texttt{race} includes \texttt{Black}, \texttt{Hispanic}, \texttt{Mixed Race}, and \texttt{Non-Black or Non-Hispanic}. The variable \texttt{total\_incarcerated} represents the count of months of incacerations in 2022 for the respondent who has the crime record.

\section{Analysis}

\begin{figure}[H]
    \begin{center}
        \includegraphics[width=.85\textwidth]{incarcerations_by_racegender}
    \end{center}
    \caption{Mean Number of Incarcerations in 2002 by Race and Gender (this is the LaTeX caption, not the ggplot title)}
    \label{fig:graph}
\end{figure}

The Figure \ref{fig:graph} above is a barplot showing the mean number of incarcerations in 2002 by \texttt{race} and \texttt{gender}.

For respondents whose variable \texttt{race} is \texttt{Black}, 


\input{../tables/incarcerations_by_racegender.tex}

\input{../tables/regress_incarcerations_by_racegender.tex}

\section{The Next Steps}

\end{document}
